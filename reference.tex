\documentclass[a4paper,oneside,twocolumn]{article}
\usepackage[utf8]{inputenc}
\usepackage[english]{babel}

\usepackage[T1]{fontenc}
\usepackage[sc]{mathpazo}
\linespread{1.2}

%\usepackage{eulervm}

\usepackage[landscape,left=.5cm,right=.5cm,top=1.5cm,bottom=.5cm]{geometry}

\setlength{\columnseprule}{0.25pt}
\usepackage{fancyhdr}
\pagestyle{fancy}
\fancyhf{}
\fancyhead[RE,RO]{ACM ICPC Reference, page \bfseries\thepage}
\fancyhead[LE,LO]{Institute of Mathematics and Statistics of University of São Paulo}

\usepackage{listings}

\usepackage{sectsty}
\sectionfont{\normalsize}

\newcommand{\SECTION}[2]{\section*{#1} \addcontentsline{toc}{subsection}{#1} #2 \begin{center}\rule{400pt}{0.25pt}\end{center}}
\newcommand{\sourcefile}[1]{\begin{center}\textbf{#1}\end{center}\lstinputlisting{src/#1}}
\newcommand{\Csourcefile}[1]{\begin{center}\textbf{#1.cpp}\end{center}\lstinputlisting[language=C++]{src/#1.cpp}}

\usepackage{amsmath}
\usepackage{amsfonts}
\usepackage{amssymb}
\usepackage{amsthm}
\newcommand{\NN}{\mathbb{N}}
\newcommand{\ZZ}{\mathbb{Z}}
\newcommand{\QQ}{\mathbb{Q}}
\newcommand{\RR}{\mathbb{R}}
\newcommand{\then}{\Longrightarrow}
\newcommand{\floor}[1]{\left\lfloor#1\right\rfloor}
\newcommand{\ceil}[1]{\left\lceil#1\right\rceil}
\newcommand{\paren}[1]{\left(#1\right)}
\newcommand{\brackets}[1]{\left[#1\right]}
\newcommand{\braces}[1]{\left\{#1\right\}}
\newcommand{\abs}[1]{\left\lvert#1\right\rvert}
\newcommand{\nequiv}{\not\equiv}
\newcommand{\ds}{\displaystyle}
\newcommand{\bigO}{\mathcal{O}}
\newcommand{\norm}[1]{\abs{\abs{#1}}}
\newcommand{\degree}{\ensuremath{^\circ}}
\newcommand{\defun}[5] {
    \begin{array}{rrcl}
#1: & #2 & \longrightarrow & #3 \\
    & #4 & \longmapsto & #5
    \end{array}
}
\renewcommand{\le}{\leqslant}
\renewcommand{\ge}{\geqslant}
\renewcommand{\epsilon}{\varepsilon}

\newcommand{\stirfst}[2]{\genfrac{[}{]}{0pt}{}{#1}{#2}}
\newcommand{\stirsnd}[2]{\genfrac{\{}{\}}{0pt}{}{#1}{#2}}
\newcommand{\bell}[1]{\mathcal B_{#1}}
\newcommand{\seg}[1]{\overline{#1}}

\newcommand{\bmat}[1]{\begin{bmatrix}#1\end{bmatrix}}
\newcommand{\vmat}[1]{\begin{vmatrix}#1\end{vmatrix}}

\title{ACM ICPC Reference}
\author{Institute of Mathematics and Statistics of University of São Paulo}

\begin{document}
\scriptsize{}
\maketitle
\thispagestyle{fancy}

\tableofcontents

\section{Numbers and Number Theory}

\SECTION{Arbitrary Precision Integers}{
    \Csourcefile{bigint}
}

\SECTION{Extended Euclid's Algorithm (Bézout's Theorem)}{
    \Csourcefile{bezout}
}

\SECTION{62-bit Modular Multiplication / Exponentiation}{
    \Csourcefile{mulmod}
    \Csourcefile{expmod}
}

\SECTION{Miller--Rabin (Primality test)}{
    \Csourcefile{isprime}
}

\SECTION{Pollard's Rho (Factorization)}{
    \Csourcefile{rho}
}

\SECTION{Brent's Algorithm (Cycle detection)}{
    Let $x_0\in S$ be an element of the finite set $S$ and consider a function $f:S\to S$. Define
        \[ f_k(x) = \begin{cases}x,& k = 0 \\ f\bigl(f_{k-1}(x)\bigr),& k > 0\end{cases}. \]

        Clearly, there exists distinct numbers $i,j\in\NN$, $i\neq j$, such that $f_i(x_0) = f_j(x_0)$.

        Let $\mu\in\NN$ be the least value such that there exists $j\in\NN\setminus\{\mu\}$ such that
        $f_\mu(x_0) = f_j(x_0)$ and let $\lambda\in\NN$ be the least value such that
        $f_\mu(x_0) = f_{\mu+\lambda}(x_0)$.

        Given $x_0$ and $f$, this code computes $\mu$ and $\lambda$ applying the operator $f$
        $\bigO(\mu+\lambda)$ times and storing at most a constant amount of elements from $S$.

        \Csourcefile{brent}
}

\section{Counting}

\SECTION{Catalan Numbers} {
    $C_n$ is:
        \begin{itemize}
    \item The number of balanced expressions built from $n$ pairs of parentheses.
        \item The number of paths in an $n\times n$ grid that stays on or below the diagonal.
        \item The number of words of size $2n$ over the alphabet $\Sigma = \{a,b\}$ having an equal number of
        $a$ symbols and $b$ symbols containing no prefix with more $a$ symbols than $b$ symbols.
        \end{itemize}

    It holds that:
        \[ C_0 = 1, C_{n+1} = \sum_{k=0}^n C_k C_{n-k} \]
        \[ C_n = \binom{2n}n - \binom{2n}{n-1} = \frac 1{n+1}\binom{2n}n = \frac{(2n)!}{n!(n+1)!} \]
}

\SECTION{Stirling Numbers of the First Kind} {
    $\ds \stirfst nk$ is:
        \begin{itemize}
    \item The number of ways to split $n$ elements into $k$ ordered partitions up to a
        permutation of the partitions among themselves and rotations within the partitions.
        \item The number of digraphs with $n$ vertices and $k$ cycles such that each vertex
        has in and out degree of 1.
        \end{itemize}

    It holds that:
        \[ \stirfst n0 = \begin{cases}1,& n=0 \\ 0,& n\neq 0\end{cases},\quad
        \stirfst 0k = \begin{cases}1,& k=0 \\ 0,& k\neq 0\end{cases} \]
        \[ \stirfst nk = (n-1)\stirfst{n-1}k + \stirfst{n-1}{k-1} \]
        \[ \stirfst n1 = (n-1)! \]
        \[ \stirfst n{n-1} = \binom n2 \]
        \[ \stirfst n{n-2} = \frac 14 (3n-1) \binom n3 \]
        \[ \stirfst n{n-3} = \binom n2 \binom n4 \]
        \[ \stirfst n2 = (n-1)! H_{n-1} \]
        \[ \stirfst n3 = \frac 12 (n-1)! \left( H_{n-1}^2 - H_{n-1}^{(2)} \right) \]
        \[ H_n = \sum_{j=1}^n \frac 1j,\quad H_n^{(k)} = \sum_{j=1}^n \frac 1{j^k} \]
        \[ \sum_{k=0}^n \stirfst nk = n! \]
        \[ \sum_{j=k}^n \stirfst nj \binom jk = \stirfst{n+1}{k+1} \]
}

\SECTION{Stirling Numbers of the Second Kind} {
    $\ds \stirsnd nk$ is the number of ways to partition an $n$-set into exactly $k$ non-empty disjoint subsets
        up to a permutation of the sets among themselves. It holds that:

        \[ \stirsnd n0 = \begin{cases}1,& n=0 \\ 0,& n\neq 0\end{cases},\quad
        \stirsnd n1 = \stirsnd nn = 1 \]
        \[ \stirsnd nk = \stirsnd{n-1}{k-1} + k\stirsnd{n-1}k \]
        \[ \stirsnd nk \bmod 2 = \begin{cases}
    0,& (n-k)\&\floor{\frac{k-1}2}\neq 0 \\
        1,& \text{otherwise}
    \end{cases}, \]
        where $\&$ is the C bitwise ``and'' operator.

        \[ \stirsnd n2 = 2^{n-1} - 1 \]
        \[ \stirsnd n{n-1} = \binom n2 \]
        \[ \stirsnd nk = \frac 1{k!} \sum_{j=0}^k(-1)^{k-j} \binom kj j^n \]
}

\SECTION{Bell Numbers} {
    $\bell n$ is the number of equivalence relations on an $n$-set or, alternatively, the number of
        partitions of an $n$-set. It holds that:

        \[ \bell n = \sum_{k=0}^n \stirsnd nk \]
        \[ \bell{n+1} = \sum_{k=0}^n\binom nk \bell k \]
        \[ \bell n = \frac 1e \sum_{k=0}^\infty \frac{k^n}{k!} \]
        \[ \bell{n+p} \equiv \bell n + \bell{n+1} \pmod p \]
}

\SECTION{The Twelvefold Way} {
    Let $A$ be a set of $m$ balls and $B$ be a set of $n$ boxes. The following table provides methods
        to compute the number of equivalent functions $f:A\to B$ satisfying specific constraints.

        \begin{tabular}{|c|c||c|c|c|}
    \hline
        Balls & Boxes & Any & Injective & Surjective \\
        \hline\hline
        $\nequiv$ & $\nequiv$ & $\ds n^m$ & $\ds\frac{n!}{(n-m)!}$ & $\ds n!\stirsnd mn$ \\
        \hline
        $\nequiv$ & $\equiv$ & $\ds \sum_{k=0}^n \stirsnd mk$ & $\ds\delta_{m\leqslant n}$ &
        $\ds\stirsnd mn$ \\
        \hline
        $\equiv$ & $\nequiv$ & $\ds\binom{m+n-1}{m}$ & $\ds\binom nm$ & $\ds\binom{m-1}{n-1}$ \\
        \hline
        $\equiv$ & $\equiv$ & {\bf(*)}$\ds\sum_{k=0}^n p(m,k)$ & $\ds\delta_{m\leqslant n}$ & 
        {\bf(**)}$\ds p(m,n)$ \\
    \hline
        \end{tabular}

    {\bf(**)} is a definition and both {\bf(*)} and {\bf(**)} are very hard to compute. So do not
    try to.
}

\SECTION{Lucca's Theorem} {
    Let $n,k,p\in\NN$ and $p$ be a prime number. Then
        \[ \binom nk \equiv \prod_{j=0}^\infty \binom{n_j}{k_j} \pmod p, \]
        where $n_j$ and $k_j$ are the $j$-th digits of the numbers $n$ and $k$ in base
        $p$, respectively.
}

\SECTION{Derangement (Desarranjo)} {
    A derangement is a permutation of the elements of a set such that none of the elements appear in their original position.

        Suppose that there are $n$ persons numbered $1, 2, \hdots, n$. Let there be $n$ hats also numbered $1, 2, \hdots, n$. 
        We have to find the number of ways in which no one gets the hat having same number as his/her number. 
        Let us assume that first person takes the hat $i$. There are $n - 1$ ways for the first person to choose the number $i$. Now there are 2 options:

        \begin{itemize}
    \item Person $i$ takes the hat of 1. Now the problem reduces to $n - 2$ persons and $n - 2$ hats.
        \item Person $i$ does not take the hat 1. This case is equivalent to solving the problem with $n - 1$ persons $n - 1$ hats (each of the remaining $n - 1$ people has precisely 1 forbidden choice from among the remaining $n - 1$ hats).
        \end{itemize}

    From this, the following relation is derived:

        $$d_n = (n-1) * (d_{n-1} + d_{n-2})$$
        $$d_1 = 0$$
        $$d_2 = 1$$

        Starting with $n = 0$, the numbers of derangements of $n$ are:
        1, 0, 1, 2, 9, 44, 265, 1854, 14833, 133496, 1334961, 14684570, 176214841, 2290792932.
}

\section{Strings}

\SECTION{Knuth--Morris--Pratt Algorithm}{
    \Csourcefile{kmp}
}

\SECTION{Aho--Corasick Algorithm}{
    \Csourcefile{ahocorasick}
}

\SECTION{Suffix Array and Longest Common Prefix (DC3)}{
    \Csourcefile{suffixarray}
}

\section{Geometry}

\SECTION{Graham's Scan Algorithm}{
    \Csourcefile{graham}
}

\SECTION{Minimum Enclosing Disc}{
    \Csourcefile{mindisc}
}

\SECTION{Sweep Circle} {
    \Csourcefile{sweepcircle}
}

\SECTION{Intersections} {
    \Csourcefile{intersections}
}

\SECTION{Lines} {
    \Csourcefile{retas}
}

\SECTION{3D Geometry} {
    \Csourcefile{geometria3D}
}

\SECTION{Centroid} {
    \Csourcefile{centroide}
}

\SECTION{Voronoi} {
    \Csourcefile{voronoi}
}

\section{Graphs}

\SECTION{MaxFlow--MinCost} {
    \Csourcefile{mcmf2}
}

\SECTION{MaxFlow} {
    \Csourcefile{maxflow}
}

\SECTION{Hopcroft--Karp} {
    \Csourcefile{hopcroft}
}

\SECTION{Maximum Match (Edmonds--Karp)} {
    \Csourcefile{EdmondsMaximumMatch}
}

\SECTION{Stoer--Wagner} {
    \Csourcefile{stoerwagner}
}

\SECTION{2-satisfability} {
    \Csourcefile{2sat}
}

\SECTION{Rooted MST in digraph} {
    \Csourcefile{mstdigraph}
}

\section{Data Structures}

\SECTION{kd-tree} {
    \Csourcefile{kdtree}
}

\SECTION{Binary Indexed Tree (BIT)} {
    \Csourcefile{bit}
}

\SECTION{Longest Increasing Subsequence (LIS)} {
    \Csourcefile{lis}
}

\SECTION{Manacher's Algorithm (Palindrome Finding)} {
    \Csourcefile{manacher}
}

\SECTION{Segment Tree (with Lazy Propagation)} {
    \Csourcefile{lazypropag}
}

\SECTION{Longest Common Ancestor (LCA)} {
    \Csourcefile{lca}
}

\end{document}

